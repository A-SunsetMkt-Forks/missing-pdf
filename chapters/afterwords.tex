\chapter*{跋}
\label{afterwords}
\bookmark[dest=\HyperLocalCurrentHref, level=-1]{跋}
\addcontentsline{toc}{chapter}{跋}

\begin{multicols}{2}
  \centering
  欣闻全卷将付梓,行文却又起波澜。\par
  人工智能显身手,密码锋刃护网安。\par
  零一舞动程序现,云物交融变革瞻。\par
  四盘硬菜飨诸位,望君细品莫畏难。

  或问何能成此作,皆言只要肯登攀。\par
  基础为贵需牢靠,软件虽杂选优端。\par
  进阶技巧拿在手,超越未来起风帆。\par
  三载百八十千字,你缺计课暂收官。
\end{multicols}

以上是两首乱写的打油诗,欸嘿\emoji{face savoring food}。不过,诗中写下的事情的确是这一年以来我们遇到的最为重大的改变——我们的作品为清华大学出版社的编辑所看到,并向我们发出了出版邀请。为了充实内容,编辑建议我们增加一些关于当下热点话题的讨论,这个想法变成了如今的超越篇。但是,如各位所见,超越篇的文章的确长——了许多,于是 Hans 称之为「四盘硬菜」。不过,出版是一件非常耗时的工作,所以距离《你缺计课》摆上书架还会有相当一段时间。

在完成超越篇之后,《你缺计课》又一次进入了「暂时收官」、慢速更新状态,但我们也在持续修订不准确的内容。为了尽可能帮助大家,内容当然得力求准确无误啦!

当然,这一年中,值得讲述的事情还有不少。我们建立了交流群,汇聚来自五湖四海的读者,吹吹水也不亦乐乎;我们在不同的地方「推销」我们的作品,说不定又有人因此而获得一些知识;我们收到了许多读者的来信,其中不乏真挚而温暖的感谢,不乏饱含感慨的长篇体悟,这都令我们备受鼓舞。\CJKsout*{你缺计课的势力正在不断地扩大!}

不过,不可避免地,我们听到了一些不同的声音,但我始终信任六个字:「尽人事,听天命。」我们做好我们力所能及的事情,至于不愿意学习新知识的人,那就只能听天由命了咯~

总之,新的一年,希望《你缺计课》能够继续发光出彩;围绕在她周围的大家也能共同成长,互促进步!

\rightline{Windy}
\rightline{2025/3/13}

\begin{center}\rule{0.5\linewidth}{0.5pt}\end{center}

光阴似箭,日月如梭。眨眼间,我们就要大学毕业了,而《你缺失的那门计算机课》也已经诞生了两年有余。

在过去的一年里,我们切切实实感受到了自己的文字正在帮助到更多的人——这是一种无法言喻的快乐。我们收到了很多读者的邮件,有提出建议的,有表达感谢的,甚至还有拿着自己写的稿件来投稿的。阅读这份教程的人,有的还在读中学,有的已经在职场奔波。我在更多地方看到了本教程的链接;而最近,本教程的 PDF 版本还被一些公众号转载。这让我们感到非常意外,也非常高兴。

受限于我们的精力,《你缺计课》已经进入了一种慢速更新的状态。我们列出了一个长长的计划表,但是完成得非常缓慢。就算是回复大家的邮件,我们也常常拖延很久。但是,我们并没有放弃。我们仍然在继续写作,继续更新。我们的愿望没有改变:我们希望我们的文字能够帮助到更多的人。

在这里,我们要感谢所有支持我们的人。感谢你们的鼓励,感谢你们的支持。《你缺计课》会和大家继续走下去。

\rightline{Hans}
\rightline{2024/4/20}

\begin{center}\rule{0.5\linewidth}{0.5pt}\end{center}

时光匆匆如白驹过隙,又是一年春节来临,也是《你缺失的那门计算机课》差不多一周年。

这一年中,《你缺计课》的内容的确在增加,但我们却有时感觉力不从心,原因很简单——没活了。
的确,在此前编写这份教程时,我们尽力将自己所能考虑到的方方面面都呈现在其中,但人无完人,百密一疏,总有我们考虑不周的部分,总有我们视线以外的角度。
最近添加的\chapref{cha:recover-from-bsod}一章,还是我们在他人遇到蓝屏问题来寻求帮助时得到的灵感。
所以我们希望所有看到这份教程的人都能提出建议,但目前来看实在不多,也算一大憾事。

这一年中,我们在意想不到的地方看到了我们的作品。
无论是有人与我们联系,还是有人的文章作了推荐,都是对我们不小的鼓舞。
很多时候,我们的耳边会响起「你做这个有什么意义呢?」之类的声音,这样的问题,其实在立场上就已经偏离了。
人们所做,并不一定对所有人有「意义」,我们所追求的,应当是「造福一方人」。
但人生在世,难免希望自己所做能够被他人认可,这既是心灵的慰藉,也是价值的体现。

这一年中,「黄金精神」的内涵在我眼前呈现。
有位朋友——暂且称之为「L」——看到了我们的作品,也想写些什么。
L 说:「受此启发,我也准备写个类似的。」
我说:「哇哦。」
这是在去年 10 月发生的事。
本月月初,我向 L 问起此事进展,L 说:「暂时还没有,事情太多了。」
我唏嘘。
后来与另一位朋友 K 聊天时提及此事,K 说:「不如说,会坚持写这个的才是少之又少吧。」
我又唏嘘。
「世上无难事,只怕有心人。」俗语中的道理显而易见,持之以恒事就能成,但它却成为了如今的「黄金精神」,因为「持之以恒」四个字,说出去只需要不到两秒,但要付诸行动却很容易半途而废。

愿新的一年,我们能继续为《你缺计课》添砖加瓦。煦风骀荡,来日方长。

\rightline{Windy}
\rightline{2023 年 1 月 30 日}

\begin{center}\rule{0.5\linewidth}{0.5pt}\end{center}

一共十三章内容版本的《你缺失的那门计算机课》PDF 初版终于问世。由于我们使用了 \LaTeX{} 来制作 PDF,而原始的《Missing》文本都是使用 Markdown 撰写的,因此我们费了不少功夫将 Markdown 版本的本教程重新排版,最终才制作成了现在大家看到的 PDF。

Windy Deng 设计了《Missing》PDF 版本的 \LaTeX{} 模板,还完成了大部分章节的重新排版和转换。如果没有他的工作,这套 PDF 或许很难和大家见面。感谢 Windy!

\rightline{Hans}
\rightline{2022 年 2 月 19 日}

\begin{center}\rule{0.5\linewidth}{0.5pt}\end{center}

距离《你缺失的那门计算机课》第一篇文章(\chapref{cha:computer-and-its-components})写成已经过去一个半月了。在这四五十天的时间里,我们忙里偷闲、陆陆续续地完成了十多篇文章,大体上已经涵盖了「最基本的电脑常识和操作」(归于「基础篇」)、「常用软件的相关介绍,以及一些优良软件的推荐」(归于「软件篇」)和「一些难以归类的使用方法技巧和一些相对比较深入的内容」(归于「进阶篇」)这样三个方面的内容,总字数已经接近 6 万字。作为完全非利益相关的小项目,《Missing》能走到目前这一步,已经是非常出乎我意料了。

撰写《Missing》的背景,如\chapref*{cha:premble}所言,是有感于身边的同学中「电脑小白」的比重之高。尽管在开始编写后不久我就意识到,我们这样写教程的读者面其实相当狭窄——我们总是更注重于「解释原理」而非「教授方法」。换言之,《Missing》的文章可以说「又臭又长」。相当多的「电脑小白」并不感兴趣于了解更多的细节,他们更愿意知道「怎么做」就了事。譬如,在\chapref{cha:archive-formats-and-tools}一章,我们花了很多篇幅,以我们自认为「非常好理解」的措辞解释了「压缩文件」的内在原理。然而,很多读者可能仅仅只想知道「我应该用哪种压缩文件」这么一个简单的事。平均长度几千字的文章,并不是他们所愿意阅读的。

但是,《Missing》之所以自名《计算机课》,而不是什么《电脑操作教程》,正是因为我们相信「理解原理」是「熟练操作」的基础。我们当然可以用一句话告诉读者「无脑选择 Chrome 浏览器就对了」「去官方网站下载 Office,再随便找个激活工具激活就行了」甚至于「买电脑选 i7 的、大内存的、固态硬盘的准没错」之类的「方法论」,但那样的话,他们或许就不知道为什么「我用的好好的 360 浏览器为什么就不行」「WPS 不就是广告比较多吗,为什么要用 Office」以及「为什么大家都推荐固态硬盘,但它真的太贵了」。我们写作的初衷,是让读者能够逐渐知道「自己的每一步操作都是在干什么」——我们希望「授人以渔」,而不是「授人以鱼」。

这是我们的美好愿景,但现实很骨感:我们并非专业的作者,《Missing》也不是一份金钱或者学分驱动的作品。我们只能做到尽力去站在目标读者——也就是几乎零基础的电脑小白——的角度,去让每一句话都写得容易理解。但这样做必然是有局限的——有些东西我们觉得很容易理解,但可能写出来就未必;有些东西我们觉得非常有必要介绍,但它反而会干扰读者的思维。我们非常需要反馈——尤其是,需要我们的「目标读者」的反馈。只有了解到「读者」的真实体会,我们才可能让这份教程变得更好。

我们正在制作本教程的 PDF 版本。受限于服务器的网络条件,《Missing》的在线版本阅读起来体验并不是很好,常常出现图片加载慢甚至网站完全打不开的情况。PDF 版本的制作可能能在一定程度上缓解此问题。我们也在思考能否借助一些其他的渠道——例如,我们所在学校的相关社团——来帮助传播和优化这份教程。尽管它是兴趣使然,但我们也从心里希望我们的《Missing》能帮助到更多的人。

《你缺失的那门计算机课》的诞生离不开很多人的鼓励和支持。在这里一并感谢你们。谢谢。

\rightline{Hans}
\rightline{2022 年 2 月 12 日}