\chapter{跋}
\label{afterwords}

一共十三章内容版本的《你缺失的那门计算机课》PDF 初版终于问世。由于我们使用了 \LaTeX{} 来制作 PDF,而原始的《Missing》文本都是使用 Markdown 撰写的,因此我们费了不少功夫将 Markdown 版本的本教程重新排版,最终才制作成了现在大家看到的 PDF。

Windy Deng 设计了《Missing》PDF 版本的 \LaTeX{} 模板,还完成了大部分章节的重新排版和转换。如果没有他的工作,这套 PDF 或许很难和大家见面。感谢 Windy!

\rightline{Hans}
\rightline{2022 年 2 月 19 日}

\begin{center}\rule{0.5\linewidth}{0.5pt}\end{center}

距离《你缺失的那门计算机课》第一篇文章(\nameref{computer-and-its-components})写成已经过去一个半月了。在这四五十天的时间里,我们忙里偷闲、陆陆续续地完成了十多篇文章,大体上已经涵盖了「最基本的电脑常识和操作」(归于「基础篇」)、「常用软件的相关介绍,以及一些优良软件的推荐」(归于「软件篇」)和「一些难以归类的使用方法技巧和一些相对比较深入的内容」(归于「进阶篇」)这样三个方面的内容,总字数已经接近 6 万字。作为完全非利益相关的小项目,《Missing》能走到目前这一步,已经是非常出乎我意料了。

撰写《Missing》的背景,如\nameref{premble}所言,是有感于身边的同学中「电脑小白」的比重之高。尽管在开始编写后不久我就意识到,我们这样写教程的读者面其实相当狭窄——我们总是更注重于「解释原理」而非「教授方法」。换言之,《Missing》的文章可以说「又臭又长」。相当多的「电脑小白」并不感兴趣于了解更多的细节,他们更愿意知道「怎么做」就了事。譬如,在\nameref{archive-formats-and-tools}一章,我们花了很多篇幅,以我们自认为「非常好理解」的措辞解释了「压缩文件」的内在原理。然而,很多读者可能仅仅只想知道「我应该用哪种压缩文件」这么一个简单的事。平均长度几千字的文章,并不是他们所愿意阅读的。

但是,《Missing》之所以自名《计算机课》,而不是什么《电脑操作教程》,正是因为我们相信「理解原理」是「熟练操作」的基础。我们当然可以用一句话告诉读者「无脑选择 Chrome 浏览器就对了」「去官方网站下载 Office,再随便找个激活工具激活就行了」甚至于「买电脑选 i7 的、大内存的、固态硬盘的准没错」之类的「方法论」,但那样的话,他们或许就不知道为什么「我用的好好的 360 浏览器为什么就不行」「WPS 不就是广告比较多吗,为什么要用 Office」以及「为什么大家都推荐固态硬盘,但它真的太贵了」。我们写作的初衷,是让读者能够逐渐知道「自己的每一步操作都是在干什么」——我们希望「授人以渔」,而不是「授人以鱼」。

这是我们的美好愿景,但现实很骨感:我们并非专业的作者,《Missing》也不是一份金钱或者学分驱动的作品。我们只能做到尽力去站在目标读者——也就是几乎零基础的电脑小白——的角度,去让每一句话都写得容易理解。但这样做必然是有局限的——有些东西我们觉得很容易理解,但可能写出来就未必;有些东西我们觉得非常有必要介绍,但它反而会干扰读者的思维。我们非常需要反馈——尤其是,需要我们的「目标读者」的反馈。只有了解到「读者」的真实体会,我们才可能让这份教程变得更好。

我们正在制作本教程的 PDF 版本。受限于服务器的网络条件,《Missing》的在线版本阅读起来体验并不是很好,常常出现图片加载慢甚至网站完全打不开的情况。PDF 版本的制作可能能在一定程度上缓解此问题。我们也在思考能否借助一些其他的渠道——例如,我们所在学校的相关社团——来帮助传播和优化这份教程。尽管它是兴趣使然,但我们也从心里希望我们的《Missing》能帮助到更多的人。

《你缺失的那门计算机课》的诞生离不开很多人的鼓励和支持。在这里一并感谢你们。谢谢。

\rightline{Hans}
\rightline{2022 年 2 月 12 日}