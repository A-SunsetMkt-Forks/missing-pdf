\chapter{序}
\label{cha:premble}

按理来说,在信息时代,熟练使用电脑应该是人们的生活必备技能。

但事实并非如此——放眼身边,从校园到职场,许多人对电脑相当陌生:他们可能面对满电脑的恶意软件不知所措,可能对着别人发来的压缩文件一头雾水,可能在电脑上装了四五个浏览器、三四个杀毒软件,更可能分不清「内存」和「硬盘」……科技飞速发展,各种新技术令人应接不暇,而他们却被数字化的浪潮淹没,无所适从。

有人说,这是智能手机的普及造成的。诚然,使用电脑与使用手机相比,复杂了不止一个数量级;而随着智能手机的不断发展,人们借助一部手机就能完成许多事情,电脑似乎已经不再需要了。可是,即使多年前就有「电脑已是夕阳产业」之论断,这么久过去了,现实依然是:我们必须学会使用电脑——不说多么精通,也至少要知道「硬盘是什么」「软件怎么找怎么装」「怎么把处理压缩文件」「出现小问题怎么办」这些 21 世纪的常识;而面对人工智能、云计算、「互联网+」等前沿技术高速发展的未来,这样的「常识」只会越来越多。

中小学的《信息技术》课堂和大学的《大学计算机基础》课程本应承担起传授这些知识的责任。可惜,事与愿违——这些课堂往往让知识为考试服务,却让我们真正需要的技能缺失了:

我们可能一辈子都不会再折腾复杂到毫无意义的页眉、页脚、艺术字和电子表格的背景、纹样,也不会再接触与动画设计和网页制作相关的任何内容。但我们的一生中,必然会无数次下载安装新软件,会无数次直面软件闪退和崩溃,会无数次为自己或他人选购新电脑,也会无数次处理电脑中积累的垃圾文件,会无数次应对各种令人头痛的电脑问题;更长远来说,数字化的浪潮总会遍及生活的方方面面,无论我们接纳还是抵触,各类新兴技术与前沿应用终将时时刻刻伴你我左右,为你我所用。

这本《你缺失的那门计算机课》希望与各位一同立于浪潮之上,拥抱数字化的新时代。旅程伊始,从电脑的内部结构,到文件管理的方法技巧,再到如何安装软件、使用软件,电脑不再是陌生的朋友;深入探索,从再探熟悉的软件,到解决问题应有的思维,再到身边人不一定知道的电脑操作技巧,使用电脑时不再迷茫无措;迈向未来,从 AI 应用与生成式 AI 的魅力,到全球网络的安全与危机,再到云计算与物联网的全面铺开,未来的无限可能尽入眼帘。放松心情,保持好奇,积极探索,享受这趟轻松而多彩的旅程。

在这趟旅程中,本书会简称自己为「《你缺计课》」或者「本书」。《你缺失的那门计算机课》的起名致敬了麻省理工大学的在线课程\href{https://missing.csail.mit.edu/}{《计算机教育中缺失的一课》(The Missing Semester of Your CS Education)}。受限于作者的知识水平,文中难免会有用词不当甚至错漏的知识,望各位多多海涵,批评指正。

\hfill Hans 和 Windy

\hfill 2024 年 12 月