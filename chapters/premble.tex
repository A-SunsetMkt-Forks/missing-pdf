\chapter{序}
\label{premble}

按理来说,对于所谓「Z 世代」的年轻人,熟练地使用电脑应该是他们的生活必备技能。

但事实却出乎我们意料。据我们观察,许多同学对电脑的使用也并不熟悉,甚至可以说是陌生:
他们可能在网上被下载到各种「P2P 高速下载器」,面对着满电脑的流氓软件而不知所措;
他们可能对着别人发来的 \texttt{zip} 或 \texttt{7z} 文件一头雾水,对「压缩」和「解压缩」都不甚熟悉;
他们可能装着四五个浏览器、三四个杀毒软件,更可能分不清自己电脑的「内存」和「硬盘」……

有人说,这是由于智能手机的普及造成的。
显然,使用电脑与使用手机相比,「复杂」了不止一个数量级;而随着智能手机的不断发展,人们借助一部手机就能完成许多事情,电脑似乎已经不再需要了。
可是,尽管几年前就有人说「电脑现在已经是夕阳产业了」,但事实却是:
至少在当下,我们仍然得学会去用电脑,不说多么「精通」,但至少要能知道「软件怎么找怎么装」「出现小问题怎么办」「XX 文件怎么打开」「怎么把文件打包」等等这些「21 世纪的常识」。

中小学的《信息技术》课堂、大学的《大学计算机基础》课程本应起到教授这些知识的作用。
可惜,事与愿违——很多时候,我们在这些课堂上学到的东西,可能一辈子都用不到;真正需要学的东西,却缺失了:

我们一辈子可能也不会再尝试用 Excel 排出张三李四不及格的科目有哪些,不会再折腾复杂到毫无意义的页眉和页脚,不会再碰和动画制作和网页相关的任何内容。
但我们未来必然有无数次会需要去网上下载一个新的软件,会无数次遇到各种各样的软件错误、闪退或者崩溃,会无数次因为 Windows 更新导致这样或那样的问题,会无数次遇到电脑沾上垃圾软件而奇慢无比无法使用……

这份《Your Missing Semester of Using Computer》是一份为「电脑小白」准备的电脑操作指南。
它直译过来就是《你缺失的那门计算机课》,也可以叫做《你本应学过的计算机课》。
我们会假定读者基本不了解电脑的操作,换言之就是所谓「电脑小白」,告诉读者「电脑最好怎么用」。

这份教程会在文中简称自己为《Missing》。
《Missing》的得名参考了 MIT 的《\href{https://missing.csail.mit.edu/}{The Missing Semester of Your CS Education}》。